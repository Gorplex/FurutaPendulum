\hypertarget{index_Introduction}{}\section{Introduction}\label{index_Introduction}
Poly\-D\-A\-Q is a customizable, embeddable data acquisition device for educational use. It was originally designed for use in the Cal Poly Mechanical Engineering Thermal Laboratory, but it has been improved with the intention that it will be usable for many different applications. It can function as a data acquisition card for a small computer or a standalone datalogger. All hardware and software is open source.\hypertarget{index_Links}{}\section{Links}\label{index_Links}
\begin{DoxyItemize}
\item \hyperlink{pd_setup}{Power and Data} For Poly\-D\-A\-Q setup of power and data connections \item \hyperlink{pd_sensors}{Connecting Sensors} For information about how to connect various types of sensors \item \hyperlink{pd_channels}{Channel Commands} For tables which show the commands used to access the data channels on each Poly\-D\-A\-Q 2 board version \item \hyperlink{pd_py_gui}{The Python G\-U\-I} For a user guide to operating the Poly\-D\-A\-Q G\-U\-I, which runs on Windows$^{\mbox{T\-M}}$ , Mac$^{\mbox{T\-M}}$ , and Linux$^{\mbox{T\-M}}$  computers.\end{DoxyItemize}
\hypertarget{index_Sensors}{}\section{Sensors}\label{index_Sensors}
Poly\-D\-A\-Q's hardware features include thermocouple, voltage, strain, and acceleration inputs. Although traces are present on the board to accommodate all of these features, some features may be left off an individual board to save money and electric power. In addition, several versions of Poly\-D\-A\-Q have the same processor and run the same software but are equipped with different sets of sensors\-: \begin{DoxyItemize}
\item Four thermocouple inputs, using A\-D8494 or A\-D8495 thermocouple signal conditioners for linear amplification and cold junction compensation. \item Four voltage inputs. Some voltage inputs use a moderately high input resistance voltage divider and an op-\/amp voltage follower with level shifter to allow an input range of -\/10 to +10 volts. This scheme is similar to that used in other low-\/cost D\-A\-Q devices such as the U\-S\-B-\/600\-X series from National Instruments(tm). Other voltage inputs can be set up without voltage dividers to give 0-\/3.\-3\-V full-\/scale measurements. The A/\-D converters use a precision 3.\-3\-V reference for accuracy. \item Up to four strain gauge bridge amplifiers whose I\-N\-A122 instrumentation amplifiers allow resolution down to the microstrain in typical Wheatstone bridge applications.\end{DoxyItemize}
\hypertarget{index_Components}{}\section{Components}\label{index_Components}
Basic components common to all Poly\-D\-A\-Q 2's include the following\-: \begin{DoxyItemize}
\item An S\-T\-M32\-F4 microcontroller which controls the data acquisition process and communicates with a P\-C (if used). \item A U\-S\-B-\/serial interface which can supply power to the board through the U\-S\-B cable as well as communicate between the Poly\-D\-A\-Q and the computer. Power can also be supplied as 5 -\/ 12 volts D\-C through a standard barrel jack. \item A Bluetooth serial modem for wireless communication with laptops, tablets, or phones (if installed). \item A micro-\/\-S\-D card socket. S\-D cards of up to around 32\-G\-B capacity can be used, and data is stored in files using the old D\-O\-S low-\/level format. C\-S\-V text files are the standard file format, but other file formats can be utilized with small changes to the firmware. Using S\-D\-I\-O card access rather than the slower S\-P\-I protocol, data has been saved at up to 36\-K\-B/s in tests so far, and we expect speed improvements as the software is refined. \item Programmability (to flash updated firmware) through a 6-\/pin S\-T\-Link2 connector.\end{DoxyItemize}
\hypertarget{index_Firmware}{}\section{Firmware}\label{index_Firmware}
The Poly\-D\-A\-Q firmware is based on the S\-T\-M32 Standard Peripheral Library and Free\-R\-T\-O\-S. This software was chosen for its compatibility with software used in Cal Poly M\-E mechatronics courses, which was in turn selected as the preferred R\-T\-O\-S environment for educational use. Free\-R\-T\-O\-S isn't known as an especially high performance R\-T\-O\-S, but its internal structure is more easily understood than those of many competing products -\/ multithreading and device access are not as hidden from the application programmer as in many other R\-T\-O\-Ses, so students can learn more about many important fundamentals. Performance is not that much of an issue, especially when we have an S\-T\-M32\-F4 processor which runs at 168 M\-Hz and has hardware floating point support.

G\-U\-I applications are being written for use on major P\-C operating systems (Linux, Mac$^{\mbox{T\-M}}$ , Windows$^{\mbox{T\-M}}$ ) and if we have good luck in porting, Android$^{\mbox{T\-M}}$ . The applications are written in Python and use the Qt G\-U\-I libraries, Num\-Py, and Py\-Qwt.\hypertarget{index_Version}{}\section{Version}\label{index_Version}
This page documents Poly\-D\-A\-Q Version 2.\-1. As of this writing (March 2015), Poly\-D\-A\-Q 2 is a work in progress. Prototypes have been made and are undergoing testing, and a production run of a dozen or so boards is being produced. 